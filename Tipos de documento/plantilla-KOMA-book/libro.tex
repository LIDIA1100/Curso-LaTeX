% !TEX TS-program = pdflatex
% !TEX encoding = UTF-8 Unicode

% Plantilla de la clase `scrbook` del paquete KOMA-script
% Francisco Torralbo Torralbo
% 2019-02-27

\documentclass{scrbook}

\KOMAoptions{%
  fontsize=10pt,        % Tamaño de fuente
  paper=b5,             % Tamaño del papel
  twoside=true,         % Para imprimir a doble página
  headings=normal,      % Tamaño de letra para los títulos: small, normal, big
  parskip=half,         % Espacio entre párrafos: full (una línea) o half (media línea)
  headsepline=false,    % Una linea separa la cabecera del texto
  cleardoublepage=empty,% No imprime cabecera ni pie en páginas en blanco 
  headsepline=false,	  % Una linea separa la cabecera del texto
  chapterprefix=false,  % No antepone el texto "capítulo" antes del número
  appendixprefix=false,	% No antepone el texto "Apéndice" antes de la letra
  %listof=totoc,		    % Añade a la tabla de contenidos la lista de tablas y figuras
  index=totoc,			    % Añade a la talba de contenidos una entrada para el índice
  bibliography=totoc,	  % Añade a la tabla de contenidos una entrada para bibliografía
  BCOR=5mm,             % Reserva margen interior para la encuadernación. El valor dependerá el tipo de encuadernado y del grosor del libro.
  % DIV=15,             % Cálcula el diseño de página según ciertos parámetros. Al aumentar el número aumentamos el ancho de texto y disminuimos el ancho del margen
}

%********************************************************************
%	PAQUETES 
%********************************************************************
\usepackage[utf8]{inputenc} 			    % Codificación de caracteres
\usepackage[spanish]{babel}           % Idioma

% MATEMÁTICAS
\usepackage{amsmath, amsthm, amssymb}
\usepackage[mathscr]{eucal} %Proporciona el comando \mathscr para fuentes de tipo manuscrito en modo matemático sin sobreescribir el comando \mathcal


% TIPOGRAFÍA
\usepackage[activate={true,nocompatibility},final,tracking=true,kerning=true,spacing=true,factor=1100,stretch=10,shrink=10]{microtype}

% TABLAS Y GRÁFICOS
\usepackage{booktabs}
\renewcommand{\arraystretch}{1.5} % Aumenta el espacio vertical entre las filas de un entorno tabular

\usepackage{graphicx}
  % Carpeta donde buscar los archivos de imagen por defecto
  \graphicspath{{img/}}

% CABECERAS
\usepackage[automark]{scrlayer-scrpage}

% VARIOS
\usepackage{xspace} % Para problemas de espaciado al definir comandos
%\usepackage{showkeys}	% Muestra las etiquetas del documento. Útil para revisar las referencias cruzadas.
% \usepackage{showframe}  % Dibuja la estructura interna del documento: cabecera y pie de página, margen, área principal del texto. 
% \usepackage[a4, odd, center, pdflatex]{crop} % Permite imprimir el documento en un a4 (si el tamaño es más pequeño) mostrando unas guías de corte. Útil para imprenta.

% ********************************************************************
% ÍNDICE 
% ********************************************************************
% Usar el comando \index{} para añadir un elemento e \index{main!sub} para añadir un elementos "sub" bajo la categoría "main".
% Usar \index{termino|textbf} para dar formato al número de página (en este caso en negrita).
% Usar \index{termino|see{termino relacionado}} para crear una referencia cruzada a otro término en el índice.

% Ejemplo: \index{espacio homogéneo}, \index{superficie!mínima}

% Para generar el índice hay que compilar el documento con MakeIndex. Generalmente los editores se encargan de ello automáticamente.
\usepackage{makeidx}
%\usepackage{showidx} % Muestra en el margen del documento las entradas añadidas al índice. Útil para revisar el documento.
\makeindex

% ********************************************************************
% COMANDOS
% ********************************************************************
% Cargamos un archivo externo donde hemos incluido todos los comandos
% propios que vamos a usar en el documento.
% DEFINICI�N DE COMANDOS Y ENTORNOS

% CONJUNTOS DE N�MEROS

  \newcommand{\N}{\mathbb{N}}     % Naturales
  \newcommand{\R}{\mathbb{R}}     % Reales
  \newcommand{\Z}{\mathbb{Z}}     % Enteros
  \newcommand{\Q}{\mathbb{Q}}     % Racionales
  \newcommand{\C}{\mathbb{C}}     % Complejos

% TEOREMAS Y ENTORNOS ASOCIADOS

  % \newtheorem{theorem}{Theorem}[chapter]
  \newtheorem*{teorema*}{Teorema}
  \newtheorem{teorema}{Teorema}[chapter]
  \newtheorem{proposicion}{Proposici�n}[chapter]
  \newtheorem{lema}{Lema}[chapter]
  \newtheorem{corolario}{Corolario}[chapter]

    \theoremstyle{definition}
  \newtheorem{definicion}{Definici�n}[chapter]
  \newtheorem{ejemplo}{Ejemplo}[chapter]

    \theoremstyle{remark}
  \newtheorem{observacion}{Observaci�n}[chapter]
 	% Comandos definidos por el usuario

% ********************************************************************
% INFORMACIÓN DEL LIBRO Y EL AUTOR
% ********************************************************************
\newcommand{\myTitle}{Institutiones calculi differentialis}
\newcommand{\myName}{Leonard Euler\xspace}
\newcommand{\myDegree}{\textsc{Doctor en Matematicas}\xspace}
\newcommand{\myDepartment}{Departamento de Matematicas\xspace}
\newcommand{\myUni}{\protect{Universidad de Basel}\xspace}
\newcommand{\myLocation}{Basel, Suiza\xspace}
\newcommand{\myTime}{Febrero 2019\xspace}

% ********************************************************************
% HYPERREFERENCES
% ********************************************************************
\usepackage{hyperref}
\hypersetup{%
	% uncomment the following line if you want to have color links
    %colorlinks=true, linktocpage=true, pdfstartpage=3, pdfstartview=FitV,%
    % uncomment the following line if you want to have black links (e.g., for printing)
    colorlinks=false, linktocpage=false, pdfborder={0 0 0}, pdfstartpage=3, pdfstartview=FitV,% 
    breaklinks=true, pdfpagemode=UseNone, pageanchor=true, pdfpagemode=UseOutlines,%
    plainpages=false, bookmarksnumbered, bookmarksopen=true, bookmarksopenlevel=1,%
    hypertexnames=true, pdfhighlight=/O,%hyperfootnotes=true,%nesting=true,%frenchlinks,%
    urlcolor=webbrown, linkcolor=RoyalBlue, citecolor=webgreen, %pagecolor=RoyalBlue,%
    %urlcolor=Black, linkcolor=Black, citecolor=Black, %pagecolor=Black,%
    pdftitle={\myTitle},%
    pdfauthor={\textcopyright\ \myName, \myUni},%
    pdfsubject={},%
    pdfkeywords={},%
    pdfcreator={pdfLaTeX},%
    pdfproducer={LaTeX with hyperref and classicthesis}%
}

\begin{document}

%********************************************************************
% FRONTMATTER
%********************************************************************
\frontmatter % Desactiva la numeración de capítulos y usa numeración romana para las páginas

\pagestyle{plain} % No imprime cabeceras

% !TeX root = ../libro.tex
% !TeX encoding = utf8

%*******************************************************
% Titlepage
%*******************************************************
\begin{titlepage}
  % if you want the titlepage to be centered, uncomment and fine-tune the line below (KOMA classes environment)
  %\begin{addmargin}[-1cm]{-3cm}
  \begin{center}
    \large  

    \hfill

    \vfill

    \begingroup
    \huge{\myTitle} \\ \bigskip
    \endgroup

    \textsf{\myName}

    \vfill

    \myDegree \\ \medskip   
    \myDepartment \\                            
    \myUni \\ 
    \bigskip

    \myTime

    \vfill                      

  \end{center}  
  %\end{addmargin}       
\end{titlepage}   

% !TeX root = ../libro.tex
% !TeX encoding = utf8

%*******************************************************
% Little Dirty Titlepage
%*******************************************************
\thispagestyle{empty}
%\pdfbookmark[1]{Titel}{title}
%*******************************************************
\begin{center}
    \textsc{\myName} \\ \medskip                        

    \begingroup
        \textsc{\myTitle}
    \endgroup
\end{center}        

% !TeX root = ../libro.tex
% !TeX encoding = latin1

\thispagestyle{empty}

\hfill

\vfill

\noindent\myName: \textit{\myTitle,} \myDegree, \textcopyright\ 2019

\medskip

\noindent\textsc{Location}: 
\myLocation

%
%\medskip
%
%\noindent\spacedlowsmallcaps{Time Frame}: \\
%\myTime

\cleardoublepage% !TeX root = ../libro.tex
% !TeX encoding = utf8

%*******************************************************
% Dedication
%*******************************************************
\thispagestyle{empty}
%\phantomsection 
\pdfbookmark[1]{Dedicatoria}{Dedicatoria}

%\vspace*{3cm}
\phantom{a}
\vfill

\begin{flushright}
\itshape
Dedicatoria
\end{flushright}

\vfill

%\newpage
%\thispagestyle{empty}
%\refstepcounter{dummy}
%\phantom{a}
%\vfill
%\begin{flushleft}
%\ttfamily
%\$ sudo apt-get install thesis \\

%\end{flushleft}
%\vfill

\cleardoublepage% !TeX root = ../libro.tex
% !TeX encoding = latin1

%*******************************************************
% Foreword
%*******************************************************
\renewcommand{\abstractname}{Prefacio}
\pdfbookmark[1]{Prefacio}{Prefacio}
\addcontentsline{toc}{chapter}{Prefacio}

\begingroup
\let\clearpage\relax
\let\cleardoublepage\relax
\let\cleardoublepage\relax

\chapter*{Prefacio}
\thispagestyle{empty}

Prólogo del libro

\endgroup			


\cleardoublepage% !TeX root = ../libro.tex
% !TeX encoding = latin1

%*******************************************************
% Acknowledgments
%*******************************************************
\pdfbookmark[1]{Agradecimientos}{Agradecimientos}
\addcontentsline{toc}{chapter}{Agradecimientos}
\bigskip

\begingroup
\let\clearpage\relax
\let\cleardoublepage\relax
\let\cleardoublepage\relax
\chapter*{Agradecimientos}
\thispagestyle{empty}

Agradecimientos del libro.

\endgroup





\pagestyle{scrheadings} % A partir de ahora sí imprime cabeceras

\cleardoublepage% !TeX root = ../libro.tex
% !TeX encoding = utf8

%*******************************************************
% Table of Contents
%*******************************************************
%\phantomsection
\pdfbookmark[1]{\contentsname}{Tabla de contenidos}
\setcounter{tocdepth}{2} % <-- 2 includes up to subsections in the ToC
\setcounter{secnumdepth}{3} % <-- 3 numbers up to subsubsections
\manualmark
\markboth{\textsc{\contentsname}}{\textsc{\contentsname}}
\tableofcontents 
\automark[section]{chapter}
\renewcommand{\chaptermark}[1]{\markboth{\textsc{#1}}{\textsc{#1}}}
\renewcommand{\sectionmark}[1]{\markright{\thesection\enspace\textsc{#1}}}

%*******************************************************
% List of Figures and of the Tables
%*******************************************************
%\clearpage

%\begingroup 
    %\let\clearpage\relax
    %\let\cleardoublepage\relax
    %\let\cleardoublepage\relax
    %*******************************************************
    % List of Figures
    %*******************************************************    
    %\phantomsection 
    %\refstepcounter{dummy}
    %\addcontentsline{toc}{chapter}{\listfigurename}
    %\pdfbookmark[1]{\listfigurename}{lof}
    %\listoffigures

    %\vspace*{8ex}

    %*******************************************************
    % List of Tables
    %*******************************************************
    %\phantomsection 
    %\refstepcounter{dummy}
    %\addcontentsline{toc}{chapter}{\listtablename}
    %\pdfbookmark[1]{\listtablename}{lot}
    %\listoftables
        
    %\vspace*{8ex}
%   \newpage
    
    %*******************************************************
    % List of Listings
    %*******************************************************      
	  %\phantomsection 
    %\refstepcounter{dummy}
    %\addcontentsline{toc}{chapter}{\lstlistlistingname}
    %\pdfbookmark[1]{\lstlistlistingname}{lol}
    %\lstlistoflistings 

    %\vspace*{8ex}
                       
%\endgroup

\cleardoublepage

\cleardoublepage% !TeX root = ../libro.tex
% !TeX encoding = utf8
%
%*******************************************************
% Introducción
%*******************************************************

\begingroup 
    \let\clearpage\relax
    \let\cleardoublepage\relax
    \let\cleardoublepage\relax

\manualmark
\markboth{\textsc{Introducción}}{\textsc{Introducción}} 
%\addtocontents{toc}{\protect\vspace{\bigskip}} % to have the bib a bit from the rest in the toc
\addcontentsline{toc}{chapter}{Introducción}

\chapter*{Introducción}
 
Introducción del libro

\endgroup
\endinput


%********************************************************************
% MAINMATTER
%********************************************************************
\mainmatter % activa la numeración de capítulos, resetea la numeración de las páginas y usa números arábigos

\setpartpreamble[c][0.75\linewidth]{%
  \bigskip % Deja un espacio vertical en la parte superior
  Lorem ipsum dolor sit amet, consetetur sadipscing elitr, sed diam nonumy eirmod
  tempor invidunt ut labore et dolore magna aliquyam erat, sed diam voluptua. At
  vero eos et accusam et justo duo dolores et ea rebum. Stet clita kasd gubergren,
  no sea takimata sanctus est Lorem ipsum dolor sit amet.
}
\part{Primera parte} % Dividir un libro en partes es opcional
\setchapterpreamble[c][0.75\linewidth]{%
  \sffamily
  Lorem ipsum dolor sit amet, consetetur sadipscing elitr, sed diam nonumy eirmod
  tempor invidunt ut labore et dolore magna aliquyam erat, sed diam voluptua. At
  vero eos et accusam et justo duo dolores et ea rebum. Stet clita kasd gubergren,
  no sea takimata sanctus est Lorem ipsum dolor sit amet.\par\bigskip
}
% !TeX root = ../libro.tex
% !TeX encoding = utf8

\chapter{Primer capítulo}\label{ch:primer-capitulo}

Cita de libro \cite{Euler1982, Euler1984, Euler1985}

\index{Leonard!Euler}

\glossary{name={GPL}, description={General Public License}, key={orden}}

Lorem ipsum dolor sit amet\marginpar{\footnotesize\raggedright Esto es una nota al margen}, consetetur sadipscing elitr, sed diam nonumy eirmod
tempor invidunt ut labore et dolore magna aliquyam erat, sed diam voluptua. At
vero eos et accusam et justo duo dolores et ea rebum. Stet clita kasd gubergren,
no sea takimata sanctus est Lorem ipsum dolor sit amet.

\endinput
%------------------------------------------------------------------------------------
% FIN DEL CAPÍTULO. 
%------------------------------------------------------------------------------------

% Añadir tantos capítulos como sea necesario

\cleardoublepage\part{Segunda parte}
% !TeX root = ../libro.tex
% !TeX encoding = utf8

\chapter{Segundo capítulo}
\renewcommand{\dictumwidth}{0.45\textwidth} % Cambia el ancho de la cita
\dictum[Leonard Euler]{Mathematicians have tried in vain to this day to discover some order in the sequence of prime numbers, and we have reasons to believe that it is a mystery into which the human mind will never penetrate.}

\section{Primera sección}
Lorem ipsum dolor sit amet, consetetur sadipscing elitr, sed diam nonumy eirmod
tempor invidunt ut labore et dolore magna aliquyam erat, sed diam voluptua. At
vero eos et accusam et justo duo dolores et ea rebum. Stet clita kasd gubergren,
no sea takimata sanctus est Lorem ipsum dolor sit amet.
\endinput
%------------------------------------------------------------------------------------
% FIN DEL CAPÍTULO. 
%------------------------------------------------------------------------------------


%********************************************************************
% APPENDIX: Opcional
%********************************************************************
\appendix % Reinicia la numeración de los capítulos y usa letras para numerarlos
% !TeX root = ../libro.tex
% !TeX encoding = utf8

\chapter{Primer apéndice}

Lorem ipsum dolor sit amet, consetetur sadipscing elitr, sed diam nonumy eirmod
tempor invidunt ut labore et dolore magna aliquyam erat, sed diam voluptua. At
vero eos et accusam et justo duo dolores et ea rebum. Stet clita kasd gubergren,
no sea takimata sanctus est Lorem ipsum dolor sit amet.

\endinput
%------------------------------------------------------------------------------------
% FIN DEL APÉNDICE. 
%------------------------------------------------------------------------------------

% Añadir tantos apéndices como sea necesario 


%********************************************************************
% BACKMATTER
%********************************************************************

\backmatter % Desactiva la numeración de los capítulos

\cleardoublepage
\manualmark \markboth{\textsc{Índice}}{\textsc{Índice}} 
\setindexpreamble{Todas los números de página impresos en \textbf{negrita} hacen referencia a la definición del término. Los números de página impresos normalmente hacen referencia a las páginas donde dicho término es usado.\par\bigskip}
\printindex

%********************************************************************
% BIBLIOGRAFÍA
%********************************************************************
\setbibpreamble{Las referencias se listan por orden alfabético. Aquellas referencias con más de un autor están ordenadas de acuerdo con el primer autor.\par\bigskip}
\bibliographystyle{alpha}
\bibliography{library.bib}


\end{document}
