% !TEX encoding = UTF-8

\documentclass[11pt]{article}

\usepackage[T1]{fontenc}
\usepackage[utf8]{inputenc}
\usepackage[a4paper,margin=3cm]{geometry}

\usepackage{amsmath,amssymb}
\usepackage{booktabs}
\usepackage[table,dvipsnames*,svgnames]{xcolor}
\usepackage{listings}
\lstset{
	language=[latex]tex,
	escapeinside={\%*}{*)},		
    inputencoding=utf8,
	extendedchars=true,
	literate={á}{{\'a}}1 {í}{{\'{\i}}}1 {é}{{\'e}}1 {ñ}{{\~n}}1 {ó}{{\'o}}1 {ú}{{\'u}}1, 	
	backgroundcolor=\color{white!90!yellow},
    commentstyle=\color{green!30!black},
    keywordstyle=\color{DarkBlue},
    numberstyle=\tiny\color{gray!30!white},
    stringstyle=\color{purple!30!purple},
    basicstyle=\ttfamily\footnotesize,
    breakatwhitespace=false,
    breaklines=true,
    firstnumber=1,
    keepspaces=true,
    numbers=left,
    numbersep=5pt,
    showspaces=false,
    showstringspaces=false,
    showtabs=false,
    tabsize=2,
    frame=no}

\usepackage[final]{showexpl}

\usepackage{siunitx}

\usepackage{tgpagella,eulervm}

\usepackage[spanish]{babel}

\begin{document}

El paquete \texttt{siunitx} permite escribir unidades y números de forma correcta de acuerdo al Sistema Internacional de Unidades y de acuerdo a unas cuantas convenciones como que las unidades se escriben en texto recto para distinguirlo de las variables o tener cuidado con los espacios entre números y unidades.

Comenzamos cargando este paquete en la cabecera del documento
\begin{lstlisting}[language=tex]
\usepackage{siunitx}
\sisetup{opciones...} % Opcional
\end{lstlisting}

Este paquete define los siguientes comandos
\begin{itemize}
\item \textbackslash ang[options]\{angle\}
\item \textbackslash num[options]\{number\}
\item \textbackslash si[options]\{unit\}
\item \textbackslash SI[options]\{number\}[pre-unit]\{unit\} 
\item \textbackslash numlist[options]\{numbers\}
\item \textbackslash numrange[options]\{numbers\}\{number2\} 
\item \textbackslash SIlist[options]\{numbers\}\{unit\}
\item \textbackslash SIrange[options]\{number1\}\{number2\}\{unit\} 
\end{itemize}

\section{Números}

El comando básico para escribir números es \texttt{\textbackslash{}num\{número\}}
\begin{LTXexample}[pos=l,rframe={}]
El n\'umero $e$ vale aproximadamente \num{2.7182818} y $\left(\pi^\pi\right)^{\pi}$ vale $\num{80662.6659385}$
\end{LTXexample}
Obsérvese que se puede usar dentro y fuera del modo matemático y, en ambos casos, formatea el número de la misma forma. Además se puede usar notación exponencial, no importa el símbolo que utilicemos como separador decimal y se puede utilizar notación exponencial
\begin{LTXexample}[pos=l,rframe={}]
\num{0.123}, \num{.123}, \num{3.5d-2}, \num{123.45 x .3e3}, \num{123.04}, \num{3,5x3.5}
\end{LTXexample}

También se pueden escribir ángulos usando un número decimal o separando con punto y coma, grados, minutos y segundos con \verb|\ang{ángulo}| o \verb|\ang{grados;minutos;segundos}|
\begin{LTXexample}[pos=l,rframe={}]
Tanto en grados sexagesimales, \ang{30} o \ang{30;0;0}, como en radianes $\pi/6$,\dots
\end{LTXexample}


\section{Unidades}

El comando básico aquí es \texttt{\textbackslash{}SI\{número\}\{unidad\}} o  \texttt{\textbackslash{}si\{unidad\}}
\begin{LTXexample}[pos=l,rframe={}]
Un cuerpo con una masa de \SI{3}{\kilo\gram} tiene una velocidad inicial de \SI{30}{\meter\per\second} (tambi\'en podemos escribir la velocidad en \si{\kilo\meter\per\second})
\end{LTXexample}

La lista de unidades que se pueden usar es muy amplia, además de que se puedan definir nuevas unidades. Las básicas reconocidas en el Sistema Internacional son  las siguientes:

\begin{center}  
\begin{tabular}{@{}lll@{}}
    \toprule
      Unidad & Orden & Símbolo \\
    \midrule
      Amperio & \verb|\ampere| & \si{\ampere}   \\
      Candela & \verb|\candela| & \si{\candela}  \\
      Kelvin  & \verb|\kelvin| & \si{\kelvin}   \\
	  Kilogramo  & \verb|\kilogram| & \si{\kilogram}\\
      Metro & \verb|\metre| & \si{\metre}    \\
      Mol & \verb|\mole| & \si{\mole}     \\
      Segundo & \verb|\second| & \si{\second}   \\
    \bottomrule
  \end{tabular}
\end{center}




\section{Tablas}

El paquete siunitx añade un nuevo tipo de columnas a las tablas que permite alinear de forma automática

\begin{LTXexample}[pos=l,rframe={}]
\begin{tabular}{lS}
\toprule
$n$ & {Valores} \\ \midrule
1 &       2.3456 \\
2 &      34.2345 \\
3 &      -6.7835 \\
4 &       90.473  \\
5 & 5642.5 \\ 
6 & 1.2e3 \\ 
7& 1.0 e4 \\
\bottomrule
\end{tabular} 
\end{LTXexample}

\end{document}
