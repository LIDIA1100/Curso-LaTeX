\documentclass[a4paper, 12pt]{article}

\usepackage[utf8]{inputenc}
\usepackage[spanish]{babel}

\usepackage{backref}
\usepackage[backref=page]{hyperref}

\begin{document}

En el documento~\cite{EulerWiki}

\begin{thebibliography}{Eul85}

\bibitem[Eul]{EulerWiki}
{L}eonhard {E}uler.
\newblock \url{https://en.wikipedia.org/wiki/Leonhard_Euler}.
\newblock Recurso online. Accedido el 14 de marzo de 2019.

\bibitem[Eul82]{Euler1982}
Leonhard Euler.
\newblock {\em Commentationes mechanicae ad theoriam machinarum pertinentes.
  {V}ol. {III}}.
\newblock Leonhardi Euleri Opera Omnia, Series Secunda: Opera Mechanica et
  Astronomica, XVII. Orell F\"{u}ssli, Z\"{u}rich, 1982.
\newblock Edited and with a preface by Charles Blanc and Pierre de Haller.

\bibitem[Eul84]{Euler1984}
Leonhard Euler.
\newblock {\em Elements of algebra}.
\newblock Springer-Verlag, New York, 1984.
\newblock Translated from the German by John Hewlett, Reprint of the 1840
  edition, With an introduction by C. Truesdell.

\bibitem[Eul85]{Euler1985}
Leonhard Euler.
\newblock An essay on continued fractions.
\newblock {\em Math. Systems Theory}, 18(4):295--328, 1985.
\newblock Translated from the Latin by B. F. Wyman and M. F. Wyman.

\end{thebibliography}

\end{document}
